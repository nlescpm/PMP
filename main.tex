\documentclass[11pt]{article}
\usepackage[a4paper,margin=1in,marginparsep=0pt,marginparwidth=0pt]{geometry} %showframe
\usepackage[xetex]{graphicx} % Required for inserting images
\usepackage{eso-pic}
\usepackage{url}
\usepackage[dutch,english]{babel}
\usepackage[colorlinks=true]{hyperref}
\usepackage{xcolor}
\usepackage{tikz}
\definecolor{nlesc-blue}{RGB}{0, 157, 221}
\definecolor{nlesc-purple}{RGB}{53, 6, 57}
\definecolor{nlesc-yellow}{RGB}{255, 178, 19}

\usepackage{caption, booktabs}
\usepackage{multirow}
\usepackage{ltablex}

\usepackage{tabularray}
\UseTblrLibrary{booktabs}
\newcommand{\testtable}[0]{
\begin{booktabs}{colspec={rccc},row{odd}={blue9}}
    \toprule
    Thing & Value & Value & Value\\
    \midrule
    A & 1 & 2 & 3\\
    B & 1 & 2 & 3\\
    C & 1 & 2 & 3\\
    \specialrule{2.5pt,teal5}{1pt}{1pt}
    D & 1 & 2 & 3\\
    E & 1 & 2 & 3\\
    \bottomrule
\end{booktabs}
}

%\newcommand{\mythinline}{\noalign{\hrule height 0.8pt}}

\newcommand{\placelogo}[0]{
\AddToShipoutPictureBG{%
%  \AtPageLowerRight{\makebox[1.3\textwidth][r]{%
%    \includegraphics[scale=1]{img/a0000-img003.jpg}}}}
\AtPageUpperLeft{\raisebox{-8\baselineskip}{\makebox[\paperwidth]{\hspace{16cm}\includegraphics[scale=1]{img/a0000-img002.png}}}}
  \AtPageLowerLeft{\raisebox{2\baselineskip}{\makebox[\paperwidth]{\hspace{16cm}\includegraphics[scale=1]{img/a0000-img003.jpg}}}} 
}%
}

\usepackage{fontspec}
\defaultfontfeatures{Mapping=tex-text,Scale=MatchLowercase}
\setmainfont{Nunito}

\usepackage{titlesec}
\titleformat{\section}{\color{nlesc-blue}\fontsize{18}{20}\bfseries}{\color{nlesc-blue}\thesection}{1.5em}{}
\titleformat{\subsection}{\fontsize{16}{18}\bfseries}{\thesubsection}{1.5em}{}

%\titleformat{\section}{\fontsize{11}{15}\bfseries}{\Roman{section}}{1.5em}{USI }

\title{{\fontsize{36}{38}\selectfont \textbf{\color{nlesc-purple}Project Management Protocol of the Netherlands eScience Center}}}
\author{Authors: Netherlands eScience Center Programme Managers}
\date{Date: September 2023\\Version: 2.0}

\begin{document}

\hypersetup{,urlcolor=blue,citecolor=blue,linkcolor=blue}

\begin{titlepage}

\AddToShipoutPictureBG*{%
  \AtPageLowerLeft{\makebox[1.32\textwidth][r]{%
    \includegraphics[scale=1]{img/a0000-img001.png}}}}


        \maketitle


  
            
\begin{table}[!h]
\resizebox{1.1\textwidth}{!}{%
\begin{tabular}{llc}
\multicolumn{1}{c}{Date} & \multicolumn{1}{c}{version} & Changes   \\
1 October 2023           & 2.0 updates                 & \multicolumn{1}{l}{\begin{tabular}[t]{@{}l@{}}DT and PM mandates, Roles of Directors, PMs role in Ambition 2,\\ SS, KD and external projects, Communications role, Editorial team, \\ breakdown of hours, opportunities beyond project, \\ Technology Status Report, End report\end{tabular}} \\
23 September 2022        & 1.0 initial version         &                                                                                                                                                                                                                                                                                           
\end{tabular}%
}
\end{table}
            
  %      \vspace{0.8cm}




\end{titlepage}

\clearpage
\setcounter{tocdepth}{2}
\tableofcontents

\clearpage

\placelogo{}
    
\section{Scope and definitions}
\label{sec:scope}

\subsection{Scope}

This document is the official project management protocol for the Netherlands eScience Center. It describes all phases
of a project and the procedures required to successfully complete them.

The scope of this document is the execution of research projects awarded by the eScience Center through calls for
proposals, though other types of projects are also briefly covered. This document gives a detailed description of all
steps, both necessary and optional, that must or may be taken in the execution of projects, reflecting the so-called
\textit{project life cycle}. For each step, the document indicates the responsibilities of the project team members
(RSEs) and other eScience Center employees (e.g. Programme Managers, Finances, Directors Team) involved in the
process.

Call procedures follow a separate protocol~\cite{call-protocol-2015} and are
not covered by this document. The call procedure protocol ends with the formal awarding of projects by the eScience
Center Governing Board or the Directors Team (DT), the notification of Lead Applicants and the formalization of the
awarding by means of a \textit{toekenningsbrief} ('Awarding letter’). The
current document describes all activities that (need to) take place from that moment onwards, until the formal closing
of the project. An independent evaluation of projects, including impact, output, process and collaboration with project
partners is outside of the scope of the current document, and will be published as a separate document or a subsequent
version of this document in the future. 

This document has been approved by the DT and will be subject to evaluation and possible adaptation annually.

The structure of this document largely follows the project life cycle (see Section~\ref{sec:scope:lifecycle}); the
protocol describes activities in chronological order.

\subsection{Stakeholders}
An eScience project is a project involving the eScience Center, where responsibility is shared between different
stakeholders who each have their own roles and responsibilities during specific phases of the project.


%\resizebox{\textwidth}{!}{%
%\begin{tabular}{p{0.15\textwidth}p{0.15\textwidth}p{0.2\textwidth}p{0.2\textwidth}p{0.1\textwidth}}
\begin{tabularx}{\linewidth}{p{0.12\textwidth}|p{0.12\textwidth}|p{0.12\textwidth}|p{0.25\textwidth}|p{0.25\textwidth}}%{@{}lX@{}}
%\caption{My caption} \\
\toprule
\textbf{Stakeholder} & \textbf{Abbreviation} & \textbf{Assignment}& \textbf{Role}& \textbf{More info}\\
\midrule
\endfirsthead
\toprule
\textbf{Stakeholder} & \textbf{Abbreviation} & \textbf{Assignment}& \textbf{Role}& \textbf{More info}\\
\midrule
\endhead
\midrule
\multicolumn{5}{r}{}
\endfoot
\bottomrule
\endlastfoot  
Lead Applicant                                     & LA                    & main applicant and recipient of the grant                                                                         & primary contact for the eScience Center project, accountable for the (quality of the) scientific contribution to the project                                                                                                           & responsibilities defined in the call text, the Terms and Conditions document, and potentially a Consortium/Collaboration agreement.  \\\hline
Programme Manager                                  & PM                    & assigned by the PM team~\footnote{All PMs, led by the Programme Director, constitute the PM team.}                                                                                          & accountable for the eScience contribution to a project, responsible for realization of project results given predetermined resources and timelines, project budget holder                                                              & Full text of responsibilities available in the PM job profile document (see Section 5 for the reference) and PM mandate (Appendix~\ref{app:sec:pm-mandate}) \\\hline
Lead Research Software Engineer                    & Lead RSE              & appointed by accountable PM                                                                                       & responsible for the timely execution of the project, main contact person for the project with other stakeholders                                                                                                                       & More details on responsibilities, see the formal role description of Lead RSE (Appendix~\ref{app:sec:leadRSE}).                                          \\\hline
Research Software Engineer (assigned to a project) & RSE                   & assigned by the accountable PM                                                                                    & responsible for the timely completion of the project                                                                                                                                                                                   & All RSE activities coordinated by Lead RSE in agreement with PM.                                                                     \\\hline
Consulting Research Software Engineer              & Consulting RSE        & involved at request of Lead RSE or accountable PM                                                                 & responsible for contributing expertise to a project for a limited but predetermined time, can be involved in some of the key meetings, in addition to an expertise contribution                                                        & All activities coordinated by Lead RSE in agreement with PM in case the project team needs additional expertise.                     \\\hline
Technology Lead                                    & TL                    & assigned by the TL team~\footnote{All TLs, led by Director of Technology, constitute the TL team.}                                                                                          & acts as point of contact for Lead RSE to the TL team. Safeguards the technological aspects of a project; accountable for the quality, reuse and sustainability of the research software developed.                                     & The TLs team is responsible for internal training programme of RSEs.                                                                 \\\hline
Section Head                                       & SH                    & assigned by the PM/SH team~\footnote{All SHs, led by Executive Director, constitute the SH team.}                                                                                       & line manager of RSEs, responsible for monitoring the overall effectiveness of RSEs in bringing projects to completion; maintain overview of a research domain.                                                                         & The SH team assigns one SH to each RSE team, and the SH ensures that team keeps its capacity and planning up to date.                \\\hline
Communications                                     &                       &                                                                                                                   & advise and facilitate internal and external communications of projects, including but not limited to showcasing projects through news items, website, newsletters, social media, interviews and videos.                                &                                                                                                                                      \\\hline
Community Manager                                  & CM                    &                                                                                                                   & advise on developing outreach activities and promoting community engagement, responsible for external training programme                                                                                                               &                                                                                                                                      \\\hline
Secretary                                          &                       &                                                                                                                   & organizes formal meetings, provide with agenda and slide template, invitation text, list of participants (with emails), and timeline.                                                                                                  &  \href{mailto:secretaries@esciencecenter.nl}{secretaries@esciencecenter.nl}                                                                                         \\\hline
Programme Director                                 & PD                    &                                                                                                                   & the escalation point for PMs, the contact point of PMs to DT for project related changes that need the DT decision, approves the workshops plans (in the agreement with F\&C on the financial part). The budget holder of Acquisition. &                                                                                                                                      \\\hline
Director of Technology                             & DoT                   &                                                                                                                   & The escalation point for TLs, the contact point of TLs to DT, accountable (and responsible) for licences and Intellectual Property (IP), software sustainability budgets holder                                                        &                                                                                                                                      \\\hline
Director of Operations                             & DoO                   &                                                                                                                   & handles legal questions (e.g., contracts, Collaborative Agreements and guest agreements)                                                                                               &                                                                                                                                      \\\hline
Finance \& Control                                 & F\&C                  & part of Operations, includes Controller, and led by DoO                                                           & responsible for maintaining financial project administration in Exact                                                                                                                                                                  & \href{mailto:finance@esciencecenter.nl}{finance@esciencecenter.nl}                                                                                            \\\hline
Directors Team                                     & DT                    & comprised of DoT, DoO, General Director and PD                                                                    & approves formal decisions regarding projects (e.g., budget changes)                                                                                                                                                                    &                                                                                                                                      \\\hline
GDPR contact person                                &                       & appointed by the DT, see the intranet for contact information                                                     & consults on GDPR~\cite{GDPR} or privacy-related issues in the project                                                                                                                                                                             & The eScience Center has not appointed a Data Protection Officer. GDPR aspects must be discussed with the contact person.             \\\hline
eScience Center project team                       & eScience project team & comprises RSEs, PM and TL working on the project                                                                  & responsible for the timely completion of the project                                                                                                                                                                                   &                                                                                                                                      \\\hline
Project team                                       &                       & comprises the eScience project team, LA and their team (including team members indicated in the project proposal) & responsible for the timely completion of the project                                                                                                                                                                                   &                                                                                                                                      \\\hline
Editorial Team                                     &                       &                                                                                                                   & provides support with outreach                                                                                                                                                                                                         & The eScience Center maintains a blog, and has presence in major social networks                                                     
%\end{tabular}%
%}
\end{tabularx}


\subsection{Types of projects}
The eScience Center receives an annual budget from NWO and SURF, the larger part of which is allocated to projects
submitted by researchers working at eligible research performing organizations in the Netherlands in the form of the
in-kind provision of RSEs. Projects may also be funded from external sources (henceforth referred to as \textit{an
external project) }or funded from the annual budget but carried out internally.

By awarding subsidy to a project or by pledging a contribution to an external project, the eScience Center takes on the
obligation to deliver high-quality work in a timely manner.

\subsubsection{Call projects}
The eScience Center publishes a range of calls. Each project is a part of a specific call (regular calls such as
OEC/ASDI, CIT/DTEC/eTEC/JEDS, SSI, or calls in collaboration with other funders such as ADAH, Big Data \& Health, GO,
JCER, eTEC-BIG, ESI-FAR). Projects from the regular calls before 2021 are partly in-cash, while projects awarded later
are fully in-kind (plus a reserved budget for workshops).

Calls can reserve part of the project or the call budget to serve the eScience Center agenda to increase the impact of
software beyond the project itself. Henceforth this will be referred to as the software sustainability budget, formerly
known as generalization budget). The budget is intended for software generalization, reuse and sustainability, and
community building. The DoT is the holder of this budget. Details concerning this budget are included in the Awarding
letter. See Section~\ref{sec:opportunities:ss} for more information.

Project teams (mainly Lead RSE, PM and TL) are expected to consult the specific call text, Awarding letter
(‘Toekenningsbrief’), Terms and Conditions document (‘Bijzondere voorwaarden’, ‘Subsidieregeling’, etc.), Consortium/Collaboration Agreement (CA), and/or
contracts for grant terms and conditions. The LA is responsible for adhering to the conditions of the project, while
the PM, with the help of the Lead RSE, monitors this.

In our call projects, most of the total requested budget is dedicated to project work and project-related activities.
The remaining part (referred to as “general activities”) covers activities that benefit our ability to contribute to
high-quality research, such as the professional development of RSEs through training, work meetings, conferences, etc,
as well as the administrative coordination and project management within the eScience Center. It is up to the PMs and
RSEs in consultation with the SHs to fairly distribute hours for general activities across all the projects they
contribute to (cf. Section~\ref{sec:exec:budget}). The exact percentage set aside for general activities is defined in
the call within which a project has been awarded.

\subsubsection{External projects}
Projects funded externally by e.g. NWO or the EU, or via private-public partnerships, are governed by external funding
conditions specified in a contract or agreement that may supersede our own rules. The budgets of these projects need to
be approved by F\&C and the DoO. Again, the project team (mostly, Lead RSE, PM, TL) must consult the specific call
text, Awarding letter, Terms and Conditions document, Consortium/Collaboration Agreements, contracts for the conditions
and rules. The LA is responsible for adherence to the rules and conditions, while the PM with the help of the Lead RSE
monitors this. Projects under external funding are covered partially by this protocol.\footnote{A budget for writing
grant proposals for external funding, use of it follows the process described in “External funding” (see Section
\ref{sec:opportunities:external-funding} for more information).}

\subsubsection{Other projects}
This document only briefly covers other types of projects such as those funded through Ambition 2, namely Dissemination
\& Community (D\&C), Knowledge \& Development (KD)\footnote{cf. Netherlands eScience Center strategy~\cite{nlesc-strategy} to get familiar with the term.} and Fellowship projects in 
Section~\ref{sec:opportunities:fellowship} and Appendix~\ref{app:pm-role}.

\clearpage
\subsection{Project life cycle overview}
\label{sec:scope:lifecycle}

\begin{figure}[!h]
    \centering
    \includegraphics[scale=0.5]{img/lifecycle-stages.png}
    \caption{Project lifecycle stages}
    \label{fig:project-lifecycle}
\end{figure}

At the eScience Center, a standard project life cycle is a three-phase process (see Figure~\ref{fig:project-lifecycle}). First, project stakeholders initiate the
project. Next, the project team executes the project and monitors its progress. Finally, once the project reaches its
end, it is formally closed.

These three phases are covered in detail in the next sections.

\clearpage
\section{Project Initiation}
\label{sec:init}
The project initiation phase starts immediately after the project has been granted. Its goal is to set up the project
within the eScience Center, including a planning in terms of staffing and a work plan.

For \textbf{external projects} and projects from specific calls (e.g., collaborative calls), F\&C ensures that all
paperwork is in place (e.g., contracts, Consortium/Collaborative Agreements, Memorandum of Understanding) before making
a project active in Exact, allowing RSEs to write hours spent on the project. The PD regularly keeps PMs up to date on
outstanding applications for external funding, signals to PMs and F\&C whenever a project has been granted, and hands
over relevant documents (such as proposal, agreements made, preliminary budget, etc) to PMs. 

PMs are accountable for call projects. For the \textbf{external projects}, PMs assign a PM and Lead RSE to the project
(i.e., the RSE involved in the submission procedure). Together with the Lead RSE the PM works with the project partners
to get all paperwork in order (such as a subcontract), cf. Section~\ref{sec:init:legal}. 

\subsection{PM assignment}
Each project has one accountable PM. The PM team assigns PMs to new projects at the first PM meeting following the
granting decision, records the assignation and asks F\&C to update Exact with new budget holder information (the newly
assigned PM). If agreement over an assignment is not reached, the PD makes the final decision in their capacity as PM
team chair.

Should the accountable PM become unavailable for an extended period, the PM team can decide to put another PM in charge
of the project.

\textbf{Responsible: PM team.}

\subsection{Administrative Setup}
The table offers an overview of the responsibilities of the different stakeholders in setting up a project:

\begin{table}[!h]
\begin{booktabs}{colspec={|>{\bfseries}p{0.15\textwidth}|p{0.15\textwidth}|p{0.65\textwidth}|},row{even}={gray!20}}
\toprule
\textbf{What} & \textbf{By} & \textbf{Responsible for}  \\[2ex]
Exact  & F\&C  &
    \begin{minipage}[t]{0.65\textwidth}
    \begin{itemize}\itemsep0em
        \item Creating project code
        \item Entering and uploading attachments
        \item Making PM budget holder 
    \end{itemize} 
    \end{minipage}  \\[2ex]
Project portfolio on SharePoint~\cite{proj-portfolio}  & F\&C  & 
    \begin{minipage}[t]{0.65\textwidth}
    \begin{itemize}\itemsep0em
        \item Creating folder in project portfolio (with template subfolder \& documents)
        \item Uploading granting package documents (incl. Awarding letter), signed start form, CA if applicable
    \end{itemize} 
    \end{minipage}  \\[2ex]
\multirow{2}{*}{Research Software Directory (RSD), project page and software pages} & PM             & \begin{tabular}[c]{@{}l@{}}Creating an RSD project page9, putting placeholder image with the eScience logo10 \\ Signalling requirements for corporate website to Communications \\ Copying website summary from the proposal (if applicable) or writing a summary and obtaining approval from LA for edited versions  \\ Finding appropriate image (e.g., royalty-free images offered by shutterstock.com and unsplash.com) \\ Ensuring Lead RSE has a maintainer access to the RSD page\end{tabular} \\
                                                                                    & Communications & Advising and reviewing content for the pages, including editing summaries, supporting with images                                                                                                                                                                                                                                                                                                                                                                                                     \\
Corporate website  & Communications &  
\begin{minipage}[t]{0.65\textwidth}
    \begin{itemize}\itemsep0em
        \item Ensuring content is displayed on the corporate web page with information supplied by PM
        \item Promoting projects to target stakeholders, when relevant. Promotion may include but is not limited to news items, inclusion in newsletters and social media
    \end{itemize} 
    \end{minipage}  \\
Ganttic  & PM             & 
\begin{minipage}[t]{0.65\textwidth}
    \begin{itemize}\itemsep0em
        \item Checking if import project information from Exact is correct 
        \item Adding labels
        \item Adding respective project portfolio URLs
        \item Planning RSEs, if applicable (e.g., for external projects)
    \end{itemize} 
    \end{minipage}  \\                    
    \bottomrule
\end{booktabs}
\end{table}


\textbf{General status and progress are monitored by the accountable PM.}

\subsection{TL assignment}
The PM asks the TL team to assign a TL to the project, providing all project information. The TL team does so at the TL
meeting and informs the PM of their decision (by assigning TL to the project in Ganttic as well as a confirmation by
email). Should the assigned TL become unavailable for an extended period, the TL team assigns another (temporary) TL to
the project.

\textbf{Responsible: TL team (at request of the PM).}

\subsection{Preparation by PM}
PM provides an overview of project requirements based on the project proposal, covering the following topics:

\begin{itemize}
\item What technology/eScience expertise is requested from the eScience Center, and at what level (novice, expert)?
\textbf{Action}: In collaboration with the TL, the PM prepares relevant tags for technologies required by the project.
\item What are the research questions and goals? \textbf{Action:} PM asks senior members of the Center with relevant domain
expertise for their opinion and prepares relevant tags for the project information.
\item What is the proposed workplan and timeline? Is the work feasible? \textbf{Action}: Together with the TL, the PM assesses
if a workplan is feasible or needs to be adjusted in the context of the Technology plan (Section~\ref{sec:init:techplan}).
\item What type of support other than RSE expertise is requested and needed? (e.g., training workshops, time and help of CMs,
use of SURF or other infrastructure) \textbf{Action}: PM notes this information for discussion with the LA and Lead
RSE. PM consults TL about management plans (see Section~\ref{sec:exec:mp}) and CMs about training workshops (see
Section~\ref{sec:exec:outreach}).
\end{itemize}
PM flags issues such as:
\begin{itemize}
\item GDPR – is there any personally identifiable information involved in the data required by the project? 
\item IP and licensing – does the project team ask for an exception to the Apache 2.0 and the CC by 4.0 default? 
\item Long term sustainability of the software – does the project have a sustainability plan? 
\item Anything else potentially problematic – for example, military application, animal or human tests, etc. (see also the
final statements in the application form).
\end{itemize}

Depending on the issue, PM contacts relevant consultants (see Section~\ref{sec:exec:consult}).

PM records all relevant information in the project log (Section~\ref{sec:exec:log}).

\textbf{Responsible: accountable PM}



\subsection{Staffing}
PM assigns the project to a team and appoints a Lead RSE in agreement with SHs and budget
holders relevant to the team activities. The PM can adjust staffing at any point during the project life cycle
whenever necessary.

PMs normally assign a project in such a way that it does not involve more than one team, unless no team is willing to
take up the project. The Lead RSE is the primary contact for the project.

PMs assign RSEs to projects, taking the RSEs' expression of interest (see Section~\ref{sec:init:vacancy} below) into account, and following due consultation with relevant stakeholders such as SHs,
TLs, RSEs and teams. 

PM communicates staffing decisions to the LA. 

\subsubsection{Project vacancy announcement}
\label{sec:init:vacancy}
Project vacancies are announced internally at the discretion of the \textbf{accountable PM} in a timely manner,via email. An announcement message must contain an instruction on how to access information on the project and how to
express interest (filling out form, via email, comments on Announcement Board in Teams etc.). In turn, RSEs express
their interest (also on behalf of their team) within the allocated time and provide a motivational text including
expertise and skills relevant to the project work. PM informs RSEs on staffing results.

If there is a shortage of RSE expertise and the project cannot be staffed, the PM signals the vacancy to the respective
SH and the PM representative in the hiring committee following rules described in the hiring process~\cite{hiring-intranet}.


\subsubsection{Assignment of RSEs}
To find RSEs suitable for the project, the PM:

\begin{itemize}
\item reviews the RSE expressions of interest,
\item checks availability of RSEs,
\item consults other PMs and the relevant SHs and TLs.
\end{itemize}

The PM assigns RSEs based on RSEs' expressions of interest, availability, technological skills and
disciplinary match. If a team of RSEs is assigned to a project, the PM and the team agree as to which member(s) and in
which capacity they work on the project. Team members are free to distribute the workload, but the Lead RSE role cannot
be freely transferred. Although each project has only one Lead RSE, team members are collectively responsible for all
the projects assigned to them, and are expected to help each other towards the successful completion of their
projects.

The Lead RSE plays a leading role in the project execution phase. The PM assigns the Lead RSE in consultation with the
relevant SH, based on (amongst others) seniority and/or potential. The PM consults the relevant SH regarding the
professional and/or personal development needs of the Lead RSE.

The Lead RSE and PM use email for all official correspondence with the project team (including the LA), keeping each
other in CC. This includes information regarding any significant change concerning the project (budget, deliverables,
changes in research team), agreements on management plans (DMP, SMP), workshops, review meetings and end reports.

\subsubsection{Lead RSE availability}
If the Lead RSE has limited availability during the project for an extended period, this is signalled to the PM and the
SH by the Lead RSE. The PM discusses with the SH whether the Lead RSE should be temporarily or permanently replaced.
The eScience project team puts forward a candidate to take up the role of Lead RSE.

The PM approves replacement of a Lead RSE. In normal circumstances, the former Lead RSE organizes a transfer meeting
with the new Lead RSE and the PM, and reports on the status of the project (current workplan, tasks, responsibilities
of all project RSEs and the next steps in the project execution), ensures proper RSD pages handover. The PM
communicates the change to the LA (or Consortium for external projects) and includes both the former Lead RSE and the
new Lead RSE in the correspondence.

\subsubsection{Project Planning}

Project planning in Ganttic is used as an agreement between the PMs and the RSEs. RSEs are expected to adhere to the
planning as agreed; if required, they can discuss and renegotiate the planning with the PMs.

To plan projects, PMs rely on information available in Ganttic. SHs ensure that information on the availability of RSEs
for at least the next 6 months is up to date; this includes the overall planning overview of an
RSE's activities, such as trainings, extended leave, etc. Other work done by RSEs (Dissemination
\& Community; Knowledge Development) are filled in by the respective budget holders. The PM (in consultation with F\&C)
ensures the planning leads to the project staying within its budget.

To make planning of projects more robust, PMs schedule RSEs on a yearly or quarterly rather than a monthly/weekly/daily
timescale, on the assumption that project hours are spent at a constant rate throughout all projects. In collaboration
with SHs and relevant budget holders, the PMs ensure that no RSE is planned beyond their capacity; should this be the
case, assignments are removed in agreement with the RSE and SH so that capacity is on par.

To facilitate a robust planning across projects and RSEs, PMs discuss the planning of each team at least once every
quarter, in a meeting including the SH associated with the team. The Lead RSE can propose planning of the project.

PMs share the resulting planning with the organization (e.g., through Ganttic).


\subsection{Kick-off meetings}
Once administration and staffing are finalized, the PM organizes two kick-off meetings: an administrative start meeting
introducing the eScience Center and our way of working and a project kick-off, which is focused on the project research
and workplan. The secretary can assist with organizing the meetings.

The PM archives the material used during the meeting and the meeting notes (from LA, PM or others) internally in the
project portfolio folder.

The PM can combine the two meetings, if necessary, into a single workshop-style meeting. This applies in particular to
specific categories of projects (e.g., based on a collaborative call, or an OpenSSI call). This is held at the eScience
Center office, and a suitable room is arranged by the PM.

For \textbf{external projects}, the way kick-off meetings are arranged depends on the nature of the project. The Lead
RSE attends all formal meetings of external projects. The PM joins these meetings if they deem this necessary. It is
the responsibility of the Lead RSE to keep the PM in the loop.

\subsubsection{Administrative start meeting}

\begin{table}[!h]
\begin{booktabs}{colspec={|>{\bfseries}p{0.15\textwidth}|p{0.85\textwidth}|},row{even}={gray!20}}
    \toprule
    Scheduled: &  Soon after awarding, but not before the Awarding letters have been sent and F\&C has collected all the paperwork and put it in Exact and Project Portfolio). \\[1.5ex]
    Stakeholders: & PM (chair), LA, Lead RSE (optional). The PM can involve others at their discretion. \\[1.5ex]
    Purpose: &  A procedural meeting to discuss how the cooperation on this project will be organized, administrative questions the LA may have, current availability of software and data, staffing, etc, so that problems can be caught early (e.g., licensing issues, no data, etc.). \\[1.5ex]
    Duration: & 1.5 hours \\[1.5ex]
    Location: & At the eScience Center (preferably), but can be also online. \\[1.5ex]
    \bottomrule
\end{booktabs}
\end{table}

For this meeting, the PM uses the administrative (PowerPoint) presentation~\cite{proj-templates}, ensuring that information is in the line
with the call text, and current Terms, IP policies, etc.

The agenda for this meeting covers:
\begin{itemize}
\item The eScience Center
\begin{itemize}
\item its mission, governance structure, technological expertise
\item request to sign up for the eScience Center newsletter
\item request to follow the eScience Center social media channels for latest updates.
\end{itemize}
\item Working with the eScience Center
\begin{itemize}
\item calls, collaboration, software and software quality, RSD
\item what are the roles of PM, Lead RSE, RSEs, teams and TL
\item suitable and welcoming work environment~\cite{arbo} for RSEs at the project location, including
working-on-location permit ('gastovereenkomst')
\item additional collaboration options: workshops, trainings, other calls.
\end{itemize}
\item Project life cycle
\begin{itemize}
\item workshops organized by the project
\item annual reviews, reports, payments
\item Project end
\item SURF Support for the projects (infrastructure, advisors).
\end{itemize}
\item Community and impact
\begin{itemize}
\item RSD, project pages, pitches, etc
\item publishing, blog posts, and outreach activities
\item digital skills programme
\item contributions to open and reproducible science initiatives.
\end{itemize}
\item Intellectual Property and Software Licenses
\begin{itemize}
\item publication protocol: funding acknowledgement in output is a must, RSEs are preferably co-authors
\item any deviation from the default IP policy (open source from the start, not only after release).
\end{itemize}
\item Project introduction
\begin{itemize}
\item project needs and expertise
\item Software and data readiness (Software and Data accessibility and quality checks).
\end{itemize}
\end{itemize}

The PM logs the agreements reached in the slides or the project log (Section~\ref{sec:exec:log}). The PM updates the
project log with the meeting date, stakeholders present and (link to) the agreements. The PM and LA share slides with
each other and the PM stores both slide decks (from PM and LA) and agreements in the project portfolio.

\subsubsection{Project Kick-off}

\begin{table}[h!]
\begin{booktabs}{colspec={|>{\bfseries}m{0.15\textwidth}|m{0.85\textwidth}|},row{even}={gray!20}}
    \toprule
    Scheduled: &  Around the date indicated by LA in the start form, after the administrative start meeting. \\[1.5ex]
    Stakeholders: & PM (chair), the entire project team, TL, and other relevant stakeholders (e.g., SH)  \\[1.5ex]
    Purpose: &  The project kick-off focuses on the execution of the project, on the technological requirements, scientific challenges, relevant communities, project goals and outputs. \\[1.5ex]
    Duration: & Max. 1.5 hours \\[1.5ex]
    Location: & At the eScience Center office or at the institute of the LA\footnote{Mandatory participants of a meeting be present in-person at the office, while all optional participants can also participate via video conferencing if they prefer.} \\[1.5ex]
    \bottomrule
\end{booktabs}
\end{table}

For this meeting the standard agenda is:
\begin{itemize}
\item Round of introductions (the entire project team) (10 min)
\item Project introduction and goals by LA (20 min)
\item Discussion on the workplan and any updates needed by the project team (30 min)
\begin{itemize}
\item eScience team explains the purpose of the technology plan
\item Project team in agreement with the PM and TL, decides when the technology plan should be submitted
\end{itemize}
\item Roles of the project team members in carrying out the project workplan (10 min)
\item (Updates to) Software Management Plan (SMP) and Data Management Plan (DMP) (5 min)
\item Agreements on initial project planning and deliverables (with concrete action points) (10 min)
\item Agreements on collaboration (e.g. frequency and location of project team meetings, planning days to work together and
location)
\item Any other business (5 min).
\end{itemize}

A workplan should always include a clear set of steps, divided into work packages, a detailed and realistic schedule, a
list of deliverables and management plans (see details in Section~\ref{sec:exec:mp}).

In agreement with the project team, the Lead RSE prepares the project for the code development (see details in Section~\ref{sec:exec:code}). PM logs the agreements, asks LA for the slides, and archives all of it in the project
portfolio.


\subsection{Technology plan}
\label{sec:init:techplan}

The project team submits a technology plan by the date agreed during the project kick-off, describing
\begin{itemize}
\item possible choices of the available technologies and which of them will be used for the project, and for which reason
\item the technological outcomes of the project (software and data) 
\item steps to be taken regarding reusability and adoptability, etc. 
\end{itemize}
The technology plan covers the choice of programming language(s), expected quality levels, etc. The plan should be seen
as an evolving record of the considerations and choices regarding the technology employed; it ensures that RSEs make
good use of the expertise present in the Center and that optimal choices are made throughout the project. An example of
project technology plan is included in the Appendix~\ref{app:exampleplan}.

\begin{table}[!h]
\begin{booktabs}{colspec={|>{\bfseries}m{0.2\textwidth}|m{0.8\textwidth}|},row{even}={gray!20}}
    \toprule
    Written by: &  Lead RSE, in collaboration with project team (including LA, TL), CMs, and others RSEs or colleagues (e.g. with relevant expertise on the subject), or relevant SIG. \\[1.5ex]
    Target audience: & Project team, TLs, PMs  \\[1.5ex]
    Schedule: &  %
    \begin{minipage}[t]{0.8\textwidth}
    \begin{itemize}\itemsep0em
        \item written at the start of the project work, before any software development starts,
        \item submitted to PM/TL by email before the deadline agreed during the project kick-off, 
        \item as a part of the project log (either full document in the log or a URL to it). 
    \end{itemize} 
      \end{minipage}
    \\[1.5ex]
    Approved by: & PM after due consultation of TL. \\[1.5ex]
    \bottomrule
\end{booktabs}
\end{table}

The Lead RSE is encouraged to reach out to RSEs or other colleagues who have the relevant expertise in the process of
developing the technology plan. CMs can advise on engaging the target audience with regard to software reusability and
adoptability. Since TLs are accountable for safeguarding the suitable technology in the project, the involvement of the
TL in writing the technology plan is important. Therefore, the PM must consult the TL on the technology plan, submitted
by the Lead RSE before any technological decisions are made in the project.

Upon approval of the technology plan (via email), the project team updates the management plans, if necessary. The Lead
RSE logs the decisions in the project log (see Section~\ref{sec:exec:log}) and archives emails in the project
portfolio, if necessary. The Lead RSE keeps the technology plan up to date: if it changes during the project, this
should be simply appended to the original technology plan (e.g., in a separate document or in the project log). The
Lead RSE explains why adaptations to the plan were required. The aim is to obtain a record of the lessons learned from.
beginning to end of the project, to facilitate collaboration and to document decisions in case a project must be
transferred to other RSEs due to unforeseen circumstances. The Lead RSE discusses any changes made to the technology
plan during the status update meetings (Section~\ref{sec:exec:status}).

\subsection{Legal agreements}
\label{sec:init:legal}
The eScience Center champions and supports open and reproducible science and open-source development. Lead applicants
get their projects awarded under the eScience Center's Terms and Conditions~\cite{nlesc-terms}. The default agreement forms offered by universities often contain IP related provisions that contradict
our Terms and conditions.

Regardless of the type of the project, RSEs must not sign any formal agreement document prior to consulting the PM. Such
documents include but are not limited to:
\begin{itemize}
\item Guest agreement (to get guest status at the project partner organization)
\item Data sharing agreement
\item IP related document
\item Consortium agreement
\item Collaborative agreement
\item Non-disclosure agreement
\end{itemize}

PM checks the draft agreement document and decides whether it can be signed and informs the RSE. If needed, the PM can
consult PD. The RSE archives the final signed version of the agreement in the project portfolio.

PM and PD signal to the DoO if legal advice is required. In that case, any proposed contracts are sent to the DoO by the
PM for final approval. The DoO shares the information within F\&C.

\clearpage
\section{Project execution}
Projects at the eScience Center vary in duration from 3 months to 5 years, depending on the call through which they were
granted. In all call projects, the PM (with the help of Lead RSE) monitors progress of the project and involves
relevant stakeholders whenever necessary. The Lead RSE takes on a leading role during the execution phase of the
project life cycle. The Lead RSE ensures that the Project team meetings take place on a regular basis: the frequency
may vary with the size of the team, e.g. full team meetings once per month and meeting with only with the LA and/or LA
team once in two weeks.

\subsection{Project logging}
\label{sec:exec:log}
eScience project team members routinely log important project events and agreements. The project log is placed in the
Project portfolio (the Coordinators subfolder, see Appendix~\ref{app:folders}). The Lead RSE keeps the log up to
date (see example in Appendix~\ref{app:example-log}). The project log facilitates the information flow between
different stakeholders about project activities.

The following should be included in the project log:
\begin{itemize}
\item RSD project page URL
\item important meetings, including dates, links to slides and fully written agreements/decisions
\item infrastructure used and decisions regarding infrastructure
\item output/deliverables (their URLs, or this is registered as an output in RSD)
\item participation in workshops, external events, conferences related to the project (or this is registered as an output in
RSD)
\item changes to the project team
\item records on management plan updates
\item technology plan decisions and updates
\item results of (code) reviews of the project.
\end{itemize}

Links pointing to other documents (e.g., files in the Project portfolio, project output, repositories) should be used in
the project log to improve readability of the log and avoid duplicate information.

\subsection{Status update meetings}
\label{sec:exec:status}
The PM stays informed about the status of the project and communicates with the Lead RSE on a regular basis.

\begin{table}[h!]
\begin{booktabs}{colspec={|>{\bfseries}m{0.15\textwidth}|m{0.85\textwidth}|},row{even}={gray!20}}
    \toprule
    Scheduled: &  Once every 4-6 weeks \\[1.5ex]
    Stakeholders: & PM (organizer), Lead RSE, optionally: TL\footnote{The TL participation is mandatory for the techology-oriented projects.}, other RSEs. \\[1.5ex]
    Purpose: &  Status update on the project and discussion around project management. \\[1.5ex]
    Duration: & 30 min –- 1 hour \\[1.5ex]
    Location: & In-person meeting is default. \\[1.5ex]
    \bottomrule
\end{booktabs}
\end{table}

PM and Lead RSE discuss:

\begin{itemize}
\item project status (including any changes in a project workplan)
\item technological issues, with due consultation of TL, respective SIG, or other RSEs, if necessary
\item changes in technology plan, technological choices (Section~\ref{sec:init:techplan}), management plans (Section
\ref{sec:exec:mp}). For any of these changes, TL presence is required
\item synergies with other projects in the Center
\item issues related to the budget, communication, staffing, etc.
\item knowledge development and transfer, potential for software reuse, software sustainability.
\end{itemize}

The frequency and duration of these meetings are at the discretion of the PM and depend on factors such as the
experience of the Lead RSE and the size of the team and/or the project. 

In projects that have a stronger focus on technology (such as the eTEC, CIT projects), the TL is involved in these
meetings more frequently. For some projects, update meetings can be combined (e.g., for projects within the same Call)
or organized in the context of a larger meeting (such as a SIG on a relevant topic). Together with the Lead RSE, the PM
decides on the format of the status update meeting.


\subsection{Project team meetings}
To keep the entire project team informed on project progress, the Lead RSE together with the LA organizes a periodical
project meeting. The frequency and format depend on the complexity of the project and size of the project team.

\begin{table}[h!]
\begin{booktabs}{colspec={|>{\bfseries}m{0.15\textwidth}|m{0.85\textwidth}|},row{even}={gray!20}}
    \toprule
    Scheduled: &  Once every 2-6 weeks \\[1.5ex]
    Stakeholders: & Lead RSE or LA (organizer), LA (chair), RSEs and other project team members from the LA side. \\[1.5ex]
    Purpose: &  Progress update on the project by all team members. \\[1.5ex]
    Duration: & 30 min –- 1 hour \\[1.5ex]
    Location: & In-person meeting is default. \\[1.5ex]
    \bottomrule
\end{booktabs}
\end{table}

The agenda of this meeting should include:
\begin{itemize}
\item status update from all stakeholders
\item discussion of scientific progress
\item discussion of technological progress, issues and choices
\item alignment of project progress with the project workplan, and adjustment of the latter, if necessary.
\end{itemize}

\subsection{Writing hours and managing project budget}
\label{sec:exec:budget}
The PM and Lead RSE must have a firm grasp of the project budget and the project duration. This information is in the
Awarding letter, the proposal and Exact.

The project execution phase roughly consists of three parts: exploration, implementation, sustaining and dissemination.
For the call projects, the rough breakdown of work vs budget is as follows: 25\% of the time and budget goes to
exploration (including learning), 50\% of the budget is to be spend on development, and 25\% is for usage,
sustainability and dissemination activities. In addition, the budget of a project should be spent at a constant rate
during the runtime of the project. Lead RSE must discuss with the PM if project execution deviates from this plan.

\begin{figure}[!h]
    \centering
    \includegraphics[scale=0.45]{img/budget-stages.png}
    %\caption{Project's budget breakdown}
    %\label{fig:project-budget}
\end{figure}

The eScience project team (including PM and RSEs) must submit their project hours in Exact by the end of each month. For
regular call projects, RSEs can write hours on awarded projects as soon as they are active in Exact, which in general
happens within a month of the project being granted. The PM writes management hours on the project budget.

Project hours are managed by different parties with different responsibilities:
\begin{table}[!h]
\renewcommand{\arraystretch}{1.5}
\begin{booktabs}{colspec={|p{0.2\textwidth}|p{0.4\textwidth}|p{0.4\textwidth}|},row{even}={gray!20}}
    \toprule
    \textbf{Stakeholder} &  \textbf{Responsibilities} & \textbf{More info} \\\toprule
    PM & 
    \begin{minipage}[t]{0.4\textwidth}
    \begin{itemize}\itemsep0em
        \item checks and approves the hours submitted in Exact for the projects for which they are accountable before the 5th workday of the next month
        \item provide monthly hour status to the Lead RSE 
        \item checks and signals to F\&C if there is an issue (for example, if project budget is incorrect)
        \item informs F\&C about project budget changes made by PM or DT
    \end{itemize} 
      \end{minipage}
    & \\\midrule
    Lead RSE &     
    \begin{minipage}[t]{0.4\textwidth}
    \begin{itemize}\itemsep0em
        \item monitors project hour expenditure and signals deviation from the workplan to the PM
    \end{itemize} 
      \end{minipage}
    &  Asks PM for hours status in Exact, or checks monthly budget status via Ganttic  \\\midrule
    F\&C &
    \begin{minipage}[t]{0.4\textwidth}
    \begin{itemize}\itemsep0em
        \item maintains accurate budget information 
        \item monitors and processes approved project hours
        \item makes financial information available to the budget holders (including PMs) each month 
        \item automatically puts read-only status on the project when project hours are depleted/exceeded early
    \end{itemize} 
      \end{minipage}  
    & All budget changes require a PM and DT decision. \\\midrule
    DoT & 
    \begin{minipage}[t]{0.4\textwidth}
    \begin{itemize}\itemsep0em
        \item monitors and approves software sustainability hours 
    \end{itemize} 
      \end{minipage}
    & There is a separate protocol on handling software sustainability. \\
    \bottomrule
\end{booktabs}
\end{table}

Hours must be submitted and approved on time, preferably on a weekly basis. All data must be entered no later than one
working week following the end of the month.

In some projects, the LA and RSEs are free to spend the awarded hours faster than originally planned. However, the Lead
RSE is responsible for results being delivered on time for the project and not exceeding the budget, and for timely
informing the PM. If a project budget is fully spent ahead of its schedule, the PM asks F\&C to restrict writing on the
project budget only to the PM.

It is possible to travel for a project, however RSEs must ask approval of their line managers (a respective SH) before
committing to an event requiring travel and fill out a travel form. See the Intranet for more information.

For \textbf{external projects}, the rules on writing hours differ from regular call projects. Only RSEs and the PM
working on the project are allowed to write hours on the project budget. Moreover, for some projects (e.g., Horizon
Europe projects) only direct contributions to the project are allowed; other non-project related activities such as
time spent on a SIG cannot be declared on these projects. The Lead RSE and PM are aware of restrictions related to
their project.

\subsection{Workshops}
\label{sec:exec:workshops}

Some call projects require the LA to organize workshops. These workshops aim at fostering research communities around
the software we develop on projects. A focus of a workshop could be, for instance, the early adoption of software
requirements suggested by potential users, the addition and expansion of new features, or the linking of the exiting
tool to a wider or more mature community. Based on the call text, the PM determines whether one of more workshops must
be organized. There are two types of workshops, namely,
\begin{itemize}
\item organised by the LA
\item organized by the Lorentz Center.
\end{itemize}

For workshops organized by the LA themselves, the LA writes and submits the workshop plan(s) to the PM (cf.~\cite{proj-templates} 
for the templates) in a timely manner. The plans need to be approved by the PD and F\&C. The
Lead RSE is expected to contribute to the workshop and its organization. A CM can advise the LA (via the Lead RSE) on
supporting the engagement and growth of relevant communities around the software; a CM is involved in the introductory
part of the workshop, including an opportunity to address participants. F\&C supports the PM team with respect to
reimbursement of the costs and records that a workshop took place. Detailed procedure of workshops approval is
described in Appendix~\ref{app:workshops}.

For workshops organised by the Lorentz Center, the LA must apply through the Lorentz Center webpage\footnote{The
procedure is described in the eScience - Lorentz Center Agreement and Annexes {\textcolor{red}(add the link when signed)}.}.

The LA has the leading role in the application. The Lead RSE takes an advisory role in the writing and design of the
workshop proposal and is expected to actively participate during the workshop event. The PM must ensure enough hours
are allocated for the Lead RSE (or RSE from the project team) to help the LA with the proposal and attend the workshop,
if they want to take an active role.

These workshops differ from the eScience-Lorentz Center Competition workshops that are funded by both the eScience and
the Lorentz Center. Besides co-funding workshop expenses, the eScience Center supports such a workshop with an
additional in-kind contribution. The role of the eScience RSEs is described in the awarded proposals. The project
initiation and assignment of the eScience project team proceeds similar to other type of project. The eScience team
clarifies with the organizers the planning and outcome of the workshop (e.g. research paper, white paper, software
release, consortium creation for a grant application, etc.). The Lead RSE has a pivotal role in the preparation and
delivery of the workshop, and, possibly, in post-workshop stage to publish any outcome drafted during the event. 

\subsection{Data and Software Management Plans}
\label{sec:exec:mp}
For some projects, Data and Software Management plans (DMP~\cite{dmp-guide} and SMP~\cite{smp-guide}, respectively) provide details regarding the
maintenance of the data and software output of the project.

Depending on the call, the LA must provide a fully worked out SMP and DMP within the first 6 months of the project. The
Lead RSE can assist the LA and their team in drafting the DMP. The LA submits the DMP to the PM, who asks a TL for
review and approval. For call projects since 2021, an SMP is a part of the proposal itself. 


The LA maintains these plans and communicates any changes to the PM and TL via the Lead RSE. If needed, the PM requests
an update. The Lead RSE can help the LA to update the plans.


\subsection{Knowledge transfer}

To increase visibility of the project and its results, the project team (including RSEs, PM, TL), Communications, CMs,
share knowledge and outcomes both inside and outside of the organization. The Lead RSE ensures that
\begin{itemize}
\item information on project results is properly shared with Communications, CMs and relevant SH in a timely manner;
\item project and software pages on the RSD are properly updated; and
\item specific requests to facilitate project visibility are sent to Communications by RSEs.
\end{itemize}

Moreover, PMs, Lead RSEs and TLs work together to spot opportunities for cross project collaboration (e.g., by reusing
software or knowledge in these projects or as a new reusability project, read more in Section~\ref{sec:opportunities:ss}).



\subsubsection{Output management}
\label{sec:exec:output}
Projects deliverables include output such as research articles, presentations, invited talks, posters, tutorials,
datasets, blog posts, white papers and workshops but also more software-oriented output types such as software or code
releases, dedicated software publications, software demonstrators, software videos, tutorials and training material
around software.

RSEs strive to apply FAIR principles~\cite{fair-principles,FAIR4RS} to all project deliverables. Therefore, all project deliverables should have

\begin{itemize}
\item concept DOIs (obtained from the publisher or created by uploading to Zenodo, arXiv, DANS or similar open-access
archives);
\item acknowledge the eScience Center project grant; and
\item listed RSEs working on the project as (co-)authors.
\end{itemize}

To formally record results and facilitate knowledge transfer, RSEs must make all project output available in the
relevant systems, online locations and databases:

TODO

In terms of project output, the PM expects the project team to follow the plan on deliverables; project deliverables are
described in the proposal and workplan. RSEs contribute to the publications and software/data releases.

Open access publications and open software are a requirement for all call projects; PM and Lead RSE keep the LA informed
on this matter, if necessary. The funds for open access publication fees are internally budgeted in the call budget by
F\&C (with the approval of the PD every year). The Lead RSE and PM consult F\&C regarding payments for an open access
publication.

For \textbf{external projects} the expected deliverables are also part of the formal project documents (proposal,
contract, etc. The Lead RSE is expected to keep the PM informed of the status of deliverables throughout the project.


\subsubsection{Outreach}
\label{sec:exec:outreach}
The Lead RSE stimulates and promotes the visibility of the project through project demonstrators, presentations, and
other means. All RSEs are expected to communicate about the project and its deliverables externally, as
described in Section~\ref{sec:exec:output}. Communications advises and supports the project team to
highlight projects through various communications channels, including but not limited to news items, social media
posts, videos and interviewing team members about relevant scientific output and impact obtained as a result of the
research software applied or developed in the project). The Lead RSE (or in rare occasion the PM) contacts
Communications with relevant information. 

Blog posts are an optional but highly recommended output of the projects. They can be authored by any member of the
project team, from LA to RSE. Topics for a project's blog post can vary greatly. Examples
include, but are not limited to,
\begin{itemize}
\item a simplified version of (parts of) the research, or 
\item a tutorial about a skill or technology a member of the team has learned during the project
\item a communication about workshops, publications, releases or other type of project output.
\end{itemize}

RSEs engage in activities to inform colleagues about the project and the results (e.g., technology plan, milestones,
code releases), including presentations at SIGs. For internal and external events, each RSE should prepare a
three-slide presentation or a pitch~\cite{three:slides} 
regularly update it. The Lead RSE is responsible for ensuring a presentation in the form of a demonstrator (e.g., of the
software developed in the project) is available after the first major release of the software.


The LA and their team are encouraged to participate in relevant Digital Skills Workshops~\cite{digital-skills} from the eScience Center. Furthermore, if the LA and their team
require a project-specific training workshop, the Lead RSE involves
\begin{itemize}
\item workshop coordination (via CMs) who can advise the eScience project team with workshop organization and the development
of new training material. If applicable, the payment for the overall organization (either through or on top of the
project budget) is handled by F\&C; and
\item the PM to discuss the RSE hours spent on organizing the training. If necessary, the PM contacts F\&C for a consult.
\end{itemize}

Again, for \textbf{external projects} separate agreements may exist with the project consortium on how to communicate
results of the project (for example, Non-Disclosure Agreement or NDA). The Lead RSE and PM consult these agreements at
the start of project to know what can be communicated and what not.

\subsection{Code quality and sustainability checks}
Taking care of software quality and sustainability is integral to the code development process cycle at the eScience
Center. All RSEs must follow our guide and best practices~\cite{guide-nlesc} for software development. Code should be made
as generic and reusable as possible from the start.

\subsubsection{Code development}
\label{sec:exec:code}
At the initial stage of code development in the project, the Lead RSE together with RSEs:

\begin{itemize}
\item set up a GitHub organization for the project, following the eScience Center Guide and the Turing Way~\cite{the_turing_way-2023}
\item add the URL of this organization to the RSD project page.
\end{itemize}

RSEs must always ask at least one project team member, relevant SIG member or other RSE at the eScience Center to
review, comment and approve pull requests in the project codebase.

\subsubsection{Code review}
As part of the annual review (see Section~\ref{sec:exec:annual}), a code review is organized by the Lead RSE. Depending
on the project it could take the form of a reusabilithon\footnote{The term coined by the Software Sustainability SIG.
It refers to the 2-3 hours session with a group of RSEs to check usability of a software, give feedback to the
developers and to come up with recommendations to improve the software (re-)usability.}, a review of code on GitHub, or
something else entirely (format to be approved by TL). The reviewers for this process are typically other RSEs at the
eScience Center.


The goal is to review software of the project for:
\begin{itemize}
\item its usability (reproducing steps of installations, and running it on a machine/laptop)
\item overall software quality and suitability
\item whenever appropriate, correctness of code
\item adherence to the technology plan (Section~\ref{sec:init:techplan})
\item adherence to eScience Center best practices
\item opportunities for reuse of software in other projects
\item correct inclusion in output systems (Section~\ref{sec:exec:output}).
\end{itemize}

Reviewers make written suggestions for improvements, and flag major issues encountered. These issues serve as input for
the TL for the formal annual review meeting (see Section~\ref{sec:exec:annual}). These notes are stored in the project log. 

\subsection{Annual project review meeting}
\label{sec:exec:annual}
For all call projects lasting longer than one year, the PM organizes annual reviews. The details are described in the
terms and conditions document~\cite{nlesc-terms}. For the projects with the
duration of exactly one year, organizing a review meeting is at the discretion of the PM (in consultation with TL and
Lead RSE). 

A standard part of every review is a discussion and list of actions on how the results of the project will be made
reusable and sustainable (as described in the DMP and SMP), how the collaboration is going and possibilities for
follow-ups to projects.

\begin{table}[!h]
\begin{booktabs}{colspec={|>{\bfseries}m{0.2\textwidth}|m{0.8\textwidth}|},row{even}={gray!20}}
    \toprule
    Scheduled: &  Yearly \\[1.5ex]
    Stakeholders: & PM (chair), Lead RSE, LA, TL, optional: other project team members, SH \\[1.5ex]
    Purpose: &  %
    \begin{minipage}[t]{0.8\textwidth}
    \begin{itemize}\itemsep0em
        \item to ensure that the project is still on track
        \item to discuss any persistent issues to ensure optimal collaboration between project team
        \item to explore opportunities beyond the project  
    \end{itemize} 
      \end{minipage}
    \\[1.5ex]
    Outcomes/Actions: & List of agreements, action points, advises (from the PM and the TL) for project team members on future steps. \\[1.5ex]
    Duration: &  Max 1.5 hours \\[1.5ex]
    Location: &  At the eScience Center or the project location\footnotemark{} \\[1.5ex]
    \bottomrule
\end{booktabs}
\end{table}
\footnotetext{Mandatory participants of a meeting be present in-person at the office, while all optional participants
can also participate via video conferencing if they prefer.}

The agenda of this meeting is:

\begin{itemize}
\item Introduction by eScience Center (round of introductions and purpose of meeting)
\item project overview and deliverables so far
\begin{itemize}
\item status of the scientific goal(s) by the LA and their team
\item status of the output (e.g., publications, software, datasets, methods, documentation) by the project team
\end{itemize}
\item status of the collaboration (including admin status of hours, bottlenecks)
\item use of digital infrastructure and support of SURF (if applicable)
\item next steps
\item opportunities beyond the project.
\end{itemize}

The goals of the meeting are to:
\begin{itemize}
\item review the progress of the project in comparison to the original workplan (are we on track?)
\item discuss research results and their novelty and current deliverables of the project
\item discuss status and update the management plans, if necessary
\item discuss strategies to expose project results to a broader community
\item discuss strategies and actions to ensure the reuse and sustainability of the software
\item identify bottlenecks and areas for improvement to ensure efficient work of the project team
\item report financial status of the project (RSE hours left)
\item brainstorm on further collaboration and funding options, if relevant
\item brainstorm on the potential for cross project collaboration.
\end{itemize}

For \textbf{external projects}, review meetings are usually organized as part of the project process. Whether or not (a
lightweight version of) our internal review procedure is needed for a project is determined by the PM, in consultation
with the Lead RSE.


\subsubsection{Review meeting preparation}
\begin{table}[h!]
  \renewcommand{\arraystretch}{1.5}
\begin{booktabs}{colspec={|>{\bfseries}m{0.15\textwidth}|m{0.85\textwidth}|},row{even}={gray!20}}
    \toprule
     &  \textbf{Stakeholder} \\
    Prepared by: & LA and Lead RSE of the project. Other stakeholders of the project can contribute. \\
    Reviewed by: &  PM accountable for project, TL accountable for technology. \\
    Target audience: PMs, TLs, SHs and RSEs. \\
    \bottomrule
\end{booktabs}
\end{table}

The PM sends the LA team the standard review meeting presentation template~\cite{proj-templates}, updating the slide on RSE hour status. This
template provides a list of the important points to be discussed. LA and Lead RSE collaboratively prepare the slides.
In particular, 

\begin{itemize}
\item the LA adds 3-5 slides (can be separate from the template) to report concisely on the extent to which the research
objectives of the project have been met. The LA is not expected to present the content of published papers or the
original workplan or proposal;
\item the Lead RSE prepares 1-2 slides on a status of the current technology plan and software/datasets/methods/documentation, 
  and remarks on reusability, adoptability and sustainability of the software;
\item the LA and Lead RSE compile the list of project deliverables. If applicable, the LA reports on the workshops;
\item the Lead RSE compiles the technology status report (see Section~\ref{sec:exec:tech}) based on the input from LA and
the rest of the project team, and provides it to the TL;
\item the LA and Lead RSE point out any scientific or technological bottlenecks, e.g., approaches that did not work, data that
was not collected, or any other reasons for delays in the workplan. To this end, the LA and Lead RSE comment if the
project is on track or whether the planning needs to be revised;
\item the PM reports on the financial status of the project (the number of hours already spent); and
\item the entire project team is invited to comment on how the collaboration is going, in terms of interaction between the
team members and suggestion for improvements, if there are any issues.
\end{itemize}


The Lead RSE:
\begin{itemize}
\item requests other project team members to contribute to the slides, wherever appropriate, and technology status report;
\item meets with CM before the review meeting to draft the technology status report, and together with TLs to draft the rest
of the report.
\item coordinates with the entire project team to finish the preparation of the presentation at least 2 working days before
the review meeting;
\item ensures that output is correctly registered in systems described in the output management (Section~\ref{sec:exec:output}) 
  and all missing URLs and DOIs are added to the slides;
\item uploads the slides to the project portfolio (into the Reviews subfolder, see Appendix~\ref{app:folders}.); and
\item informs PM and TL that slides are ready and are in the project portfolio.
\end{itemize}

\subsubsection{Technology status report}
\label{sec:exec:tech}
Prior to the annual review meeting, the PM asks Lead RSE to fill in the technology status report. This is a document
containing an overview of the technical development of the project such as project RSD page with its deliverables, URLs
to project plans, and information on quality and community involvement. The information provided by the Lead RSE (in
collaboration with the project team) in this report serves as input for the TL at the project review meeting.

Two weeks prior to the review meeting, the TL should receive from the Lead RSE the filled-out
form. Based on the final report the TL performs a code review or software health check, and ensures that the review
results are shared with the eScience project team before the annual meeting. The Lead RSE, the TL and the PM can meet
additionally to discuss the status report and/or the results prior to the annual review meeting.

The report and the review are archived by the TL in the project portfolio folder.

\subsubsection{At the review meeting}
The time breakdown of the meeting agenda is follows:
\begin{itemize}
\item presentation by the LA (max. 20 min)
\item presentation by the Lead RSE (max. 20 min), including a description of RSE roles and project deliverables.
\item discussion (max. 40 min)
\item summary, action points and conclusions.
\end{itemize}

The PM chairs the meeting, acting as a reviewer together with the TL. The TL raises possible issues related to
technology and software. Other invited stakeholders can comment and contribute to the discussions. The PM and TL
comment on the status of the deliverables:

\begin{itemize}
\item have the objectives outlined in the proposal been sufficiently addressed? (PM)
\item does the project follow the workplan in terms of deliverables? (PM)
\item has the output been registered according to the rules of output management (Section~\ref{sec:exec:output})? (PM, TL)
\item does all project output have publications (including software and data papers)? (PM)
\item are there any issues flagged during the code review that needs to be discussed with the project team? (TL)
\item does the project team sufficiently engage and align with relevant communities (e.g., via the workshops)? (PM)
\item does the project adhere to the technology plan, SMP and DMP? (TL)
\end{itemize}

The eScience project team comments on any further possibilities for reusability, adoptability and sustainability of the
software, and the project team comments on possible collaborations beyond the project.

The PM updates the slides with action points, agreements and plans (with the project partners agreement). The PM logs
the meeting in the project log.

\subsubsection{After the review meeting}
PM shares the updated slide deck with the project team members to check the agreements written down. PM ensures that the
final version of the presentation(s) uploaded to the project portfolio is correct.

\subsection{Reporting}
For call projects the annual review meeting and end report serve as formal progress reports.

\textbf{External projects} may require periodic reporting to the consortium on progress according to the workplan,
including deliverables. The PM and Lead RSE consult the Consortium/Collaboration Agreement, the contract and the
proposed workplan and involve F\&C for the financial part of the report. Normally, the external project coordinator
(e.g., EU project coordinator, NWO programme officer) signals the deadline of a deliverable or report. The Lead RSE
contributes to the report on project activities required to be done by the eScience Center, and the PM checks the
document. The PM asks F\&C to check or fill in the financial part of the report, signed by the DoO if necessary. Once
the final version is ready, the PM sends it to the external project coordinator (via EU portal done by F\&C) and
archives this report in the project portfolio.

\subsection{Conflict resolution and complaint procedure}
The eScience Center follows the Code of Conduct as outlined in the Personnel Policy (cf.~\cite{cao-intranet} for the details).

All conflicts on projects involving the eScience Center RSEs and the LA and their team should be resolved using the
following four-step process:

\begin{enumerate}
\item When problems arise in a project, the Lead RSE is expected to resolve problems in consultation with the LA. If needed,
the PM can be asked to join in discussions on finding the best course of action;
\item Both RSEs and LA can escalate issues related to the project to the PM (preferably, via the Lead RSE). The PM organizes a
meeting to discuss the problem and tries to resolve it;
\item If the problem remains unsolved, the RSEs or the LA can escalate it to the PD by sending a letter summarizing the
situation to the accountable PM, who will forward it to the PD; and
\item The PD can escalate the problem to the DT.
\end{enumerate}

If the problem is with the PM, RSEs can escalate to their manager (SH).

The eScience Center also has an external confidential advisor ('Vertrouwenspersoon') who can be contacted anonymously. See the Personnel
Policy for details.

Resolving conflicts may result in changes to the project, such as changes in staffing or changes described in Section~\ref{sec:init:vacancy}.

\subsection{Non-RSE Consultants}
\label{sec:exec:consult}
Certain issues will require that the project team consults with other persons inside the eScience Center. The following
situations require the team member to notify the PM, any actions may be delegated to any eScience team member:

\begin{itemize}
\item For project related GDPR issues, or personal identifiable data, consultation with the GDPR contact person is
obligatory. For contacts, cf. GDPR section on the Intranet~\cite{proj-intranet}. 
\item For issues related to software or data accessibility and quality, contact the TL.
\item For issues related to scientific integrity, contact the scientific integrity officer (cf. the Intranet page~\cite{proj-intranet}).
\item For issues related to SURF (use of their infrastructure or need of advisor), contact SURF liaisons. For
contacts, see the intranet page~\cite{access-infra}.
\item For issues around sustainability, contact the TL and CMs.
\item For IP and licensing issues, contact the DoT.
\item For legal matters, contact the DoO (cf. Section~\ref{sec:init:legal}).
\item For external opportunities, contact PD (cf. Section \ref{sec:opportunities:external-funding}).
\end{itemize}

\subsection{Changes to the project}
\label{sec:exec:changes}
During the project life cycle, the workplan may change substantially:

\begin{itemize}
\item New deliverables because of additional funding
\item New workplan because of changes in the research goal and/or in the technology used
\item Timeline, leading to a different end date.
\end{itemize}

Any of these changes needs explicit approval from the PM team or the DT.

\begin{table}[h!]
\begin{booktabs}{colspec={|m{0.15\textwidth}|m{0.85\textwidth}|},row{even}={gray!20}}
    \toprule
     \textbf{Type of request} &  \textbf{Decided by:} \\[1.5ex]
    Budget requests within the PM mandate (see Appendix~\ref{app:pm-mandate}.) & PM team \\[1.5ex]
    All requests regarding budget changes outside the PM mandate &  DT (via PD) \\[1.5ex]
    Early termination & DT (and DT informs the Board) \\[1.5ex]
    \bottomrule
\end{booktabs}
\end{table}


For \textbf{external projects} changes to a project must be handled as described in the formal documents for this
project (e.g., grant agreement, consortium agreement). If it is within the PM mandate, the PM discusses with the PM
team any extensions required for the project. Otherwise, the decision is made by the DT. The PM informs the external
funder or consortium of the decision.

\subsubsection{Proposal changes request}
The LA must submit a formal request to the PM team (by email via the PM, in PDF format, signed) containing:

\begin{itemize}
\item project title and project number
\item requested change (e.g., time/dates, RSE hours, scientific goal) and motivation for this change
\item conditions such as deliverables: 
\begin{itemize}
\item If there are new deliverables, what are those and what is the new planning? 
\item If there are no new deliverables, that should be stated explicitly.
\end{itemize}
\item any motivated budget change, such as 
\begin{itemize}
\item LA wants to increase their involvement 
\item change in research personnel (if applicable in the case of older projects)
\item transfer from hardware costs to RSE contribution or PYR for research personnel on the LA side (or vice versa) (if
applicable in the case of older projects)
\item any in-kind to cash change, or vice versa (including requests with the extra cash budget from the LA).
\end{itemize}
\item any prior or planned inactivity on the project, such as
\begin{itemize}
\item shortage of personnel on the LA side due to e.g. maternity leave, sick leave, hiring delays (for example, a PhD student
or a postdoc needs to be hired but there is a concise timeline on the hiring procedure)
\item unavailability of RSEs 
\item additional data that needs to be collected.
\end{itemize}
\item any delay with the start date.
\end{itemize}


\subsubsection{Processing the changes request and decision}
Upon receipt of the request, the PM assesses if the request should be granted based on considerations such as 
\begin{itemize}
\item whether the new objective is scientifically promising or technologically interesting? (if applicable)
\item collaboration status with the LA;
\item prior problems regarding the project;
\item the benefit of continuing the project for the eScience Center (e.g., good wrap up of the collaboration, this leads to
another funding opportunity together); and/ or
\item availability of RSEs with relevant expertise to work on it.
\end{itemize}

The PM can consult with RSEs and the TL on whether the new planning is feasible. In case of additional funding, the DT
(via the PD) will decide, after a budget calculation by the F\&C and approval by the DoO. Otherwise, the PM puts the
request on the agenda for the next PM meeting, containing: 
\begin{itemize}
\item the motivated request (uploaded to the project portfolio, the subfolder titled Coordinators (see Appendix~\ref{app:folders} for more details on the folder structure.})
\item the recommended action
\item the prepared decision on the PM meeting agenda (uploaded to the project portfolio, the subfolder titled Coordinators).
\end{itemize}

The PM team may request more information from the LA via the PM (and thus postpone the decision on the request). The LA
can provide the new information via an additional PDF signed letter or as amendment to the original letter.

After the final decision, the PM notes the official decision in the decision document. If the request is not approved,
the PM communicates this to the LA. If the request is approved, the PM
\begin{itemize}
\item communicates with F\&C, which finalizes the extension (changes in Exact, the extension letter for the LA);
\item double checks if budget and hours in Exact are still correct;
\item updates planning and adjusts staffing, if necessary;
\item ensures website and RSD are updated (e.g., if dates or affiliation changed); and
\item communicates the extension to the project team.
\end{itemize}


If the request involves a DT decision, the PM submits a request formally through the PD.

\subsubsection{Early project termination}
Early project termination can be
\begin{itemize}
\item agreed on by mutual consent;
\item initiated by the LA; and/ or
\item initiated by the eScience Center.
\end{itemize}

In the first and the second case, the PM submits a letter (written together with and signed by the LA) explaining the
situation to the DT. The letter should contain (a proposal for) an agreement on how to handle all the remaining
resources of the project (RSE hours, cash contribution, FTE commitment for LA, workshops, software sustainability
budget, etc.). 

The PM can request the termination of a project if the conditions and agreements in the Awarding letter and Bijzondere
voorwaarden have been violated by the LA or the project partners. The PM submits the letter to the DT (via the PD
explaining the situation). If the DT approves the termination, the PM communicates this decision to F\&C, which
finalizes the process (by making changes in Exact and preparing a termination letter).

\subsection{Opportunities beyond the project}
\label{sec:opportunities}
A project team can explore different opportunities for ideas that stem from the project that go beyond its scope and/or
budget. Appendix \ref{app:role-pm} summarizes the role of PM in such projects. The PM discusses these opportunities with the project
team during the review meeting.


\subsubsection{Increasing reusability (in this document called software sustainability)}
\label{sec:opportunities:ss}
For some projects, or entire calls, specific budget is available for software sustainability. Until 2020, each
individual project was assigned a so-called Generalization budget (“Generalisatie”), for generalization and reuse of
project results. Since 2020, however, this budget is no longer assigned per project, but for the entire programme/call.
The PM and the Lead RSE check the corresponding call text and the awarding letter of the project to determine If the
software sustainability budget is available.

The PM signals potential for reusability to the TL. The TL discusses this opportunity with the Lead RSE and if
necessary, the TL team. If RSEs have an idea and are interested to work on a project funded by this budget, they may
contact TLs or DoT for more information. The process follows the SS protocol~\cite{ss-intranet}.

\subsubsection{Knowledge and Development}
\label{sec:opportunities:kd}
For the development of broad and deep knowledge on digital technologies and their application, RSEs can apply for
so-called Knowledge and Development (KD) project. The process is described by the KD protocol~\cite{kd-intranet}.

\subsubsection{External funding}
\label{sec:opportunities:external-funding}
The project team may decide to pursue other funding opportunities. The eScience Center encourages RSEs to pursue
external funding opportunities to promote the organization profile as research organization. PD oversees the
Acquisition activities and the process. The relevant information is available via Intranet~\cite{acqisition-intranet}.


\subsubsection{Fellowship}
\label{sec:opportunities:fellowship}
To stimulate community engagement lasting longer than project lifetime, the eScience Center funds annual Fellowship
Program~\cite{fellowship-intranet}. The eScience project team is not eligible for this program, however, the LA and their team are.


\clearpage
\section{Project closing}
\label{sec:closing}

Project closing is the final phase of a project. In this phase the PM (with the help of F\&C) processes the end report
and accepts the project deliverables. Once the project is formally closed, RSEs can no longer write hours or work on
this project. 


\bigskip

\subsection{End report}
\label{sec:closing:end}
All completed call projects at the eScience Center must have an end project report.

\begin{table}[!h]
\begin{booktabs}{colspec={|>{\bfseries}m{0.2\textwidth}|m{0.8\textwidth}|},row{even}={gray!20}}
    \toprule
    Written by: &  the LA, assisted by the Lead RSE. \\[1.5ex]
    Target audience: & PMs, RSEs, Communications (layman summary), F&C (accountants), TLs \\[1.5ex]
    Schedule: &  %
    \begin{minipage}[t]{0.8\textwidth}
    \begin{itemize}\itemsep0em
        \item written in last months of the project,
        \item submitted 3 months after the project end at latest, 
        \item archived in the project portfolio on the internal All SharePoint site. 
    \end{itemize} 
      \end{minipage}
    \\[1.5ex]
    Approved by: & The PM team and F\&C. \\[1.5ex]
    \bottomrule
\end{booktabs}
\end{table}

\bibliographystyle{abbrv}
\bibliography{references}

\end{document}
