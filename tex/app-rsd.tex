\section{RSD project pages}
\label{app:rsd-page}

Each project at the eScience Center has an \href{https://research-software-directory.org/organisations/netherlands-escience-center?tab=projects}{RSD page}. 
\paragraph{Project Initiation} At the project start, the following data should be added to the RSD project page:
\begin{itemize}
   \item grant ID: currently, the Exact number of the call project or the Awarding letter reference (will be replaced by \href{https://raid.org/}{RAiD} later).
        External funders (e.g., EU, NWO) often provide their own ID.
   \item funder(s): list of all funding organizations, including the eScience Center.
   \item dates: Project start and end dates.
   \item research domain: relevant project domain, often from the proposal. Multiple domains can be chosen.
   \item participating organizations: list of the organizations, including the eScience section(s).
   \item description: Typically copied from the proposal abstract.
   \item a cover image: initially a placeholder image provided by the Communication
   \item relevant keywords and categories
   \item the call type: the specific funding scheme or program through which the project is supported.
 \end{itemize}
 When reusing a project page under new funding, Lead RSE updates it with the new funding source and details.

\paragraph{Project Execution}
During the project, the following data should be added or updated as needed:
\begin{itemize}
   \item repository link: URL to the GitHub (or other version control platform) containing the project's codebase.
   \item project team: the list of all project team members including the LAs and their team, with roles and ORCIDs where possible. 
     RSEs should update the list as the team members change, and use the role field to indicate involvement periods.
   \item related projects: the link(s) to any precursor project(s) in RSD.
   \item deliverables: any project deliverables (see also Section~\ref{sec:exec:output}) including but not limited to
  \begin{itemize}
   \item  all research deliverables of the project, i.e. publications such as papers or published datasets (using the DOI provided by the publisher), preprints of papers (using the DOI provided by the preprint server)
        self published research outputs, such as datasets, presentations, posters, reports, etc. (using the DOI it receives when uploaded to Zenodo), Online material such as websites, notebooks, online tutorials, blogs, videos, news items, interviews, etc. (registered using the URL and a short description).
   \item all software deliverables: all reusable software should have its own RSD software page and be linked as related software. Single-use software (e.g., scripts) can also be archived or released via Zenodo and can be added via DOI. Versioned releases archived in Zenodo can be added via DOI, in cases when software has been developed in multiple projects.
   \item design documents: the documents such as the Technology Plan, SMP and DMP can be published on Zenodo, and added using their DOIs.
 \end{itemize}
 \end{itemize}
 \paragraph{Project Closing}
Once the project is coming to an end, in addition to the end report data, the following data RSEs are highly encouraged to add:
\begin{itemize}
   \item testimonials from the project partners or user community
   \item list of all digital infrastructure used by the project team during the entire project
 \end{itemize}