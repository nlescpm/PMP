\section{Role of PMs and others in external and Ambition 2 projects}
\label{app:role-pm}

This section focuses speficially on the role and involvement of the PM team in Fellowship,
External, KD, SS and D\&C projects.


\subsection{Fellowship projects}
Funded by the Calls budget, the purpose and the organization of these projects are different from call projects. CMs are
responsible for Fellowship projects, and a PM assigned by the PM team is advising them.

\subsection{External projects}

\begin{enumerate}[label=\arabic*.,ref=\arabic*]
\item Acquisition
\begin{itemize}
\item The eScience Center employees signal an acquisition opportunity and must follow the policy for
external funding and the process to ask for permission. The PMs must be consulted as part of this process.
\item During this consultation, the PM needs to know about:
\begin{itemize}
\item (estimate of) the work involved in person hours:
\item composition of the project team and whether the person submitting the proposal wants to be the Lead RSE themselves
\item timeline of the proposal/project
\end{itemize}
\item The PM can advise the Lead RSE on the proposal. If this advice is negative or critical, the PM
contacts the PD. If the PM thinks that additional (temporary) capacity is needed in case the project is granted, the PM
contacts the relevant SH. Note: the planning cannot be made definitive at this stage.
\end{itemize}
\item Preparation of project – call/subsidy projects
\begin{itemize}
\item The Lead RSE informs the PD, F\&C, PM, SH and other relevant persons as soon as a granting
confirmation has been received
\item The PD, F\&C and Lead RSE are involved in the preparation stage of the project such as grant
and/or consortium agreements, starting documents, etc.
\item The PM discusses the planning with the Lead RSE and other project team members and provides a
planning in Gantic.
\end{itemize}
\item Preparation of project – contract projects
\begin{itemize}
\item The lead RSE and F\&C are in charge of arranging the contract. The lead RSE keeps the PM in
the loop about starting dates
\item The PM discusses the planning with the Lead RSE and other project team members and provides a
planning in Gantic.
\end{itemize}
\item During the project
\begin{itemize}
\item See project protocol for the relevant parts. The PM acts in consulting role and Lead RSE is
accountable.
\end{itemize}
\item Reporting
\begin{itemize}
\item See project protocol for the relevant parts. As described in Section 3.9, the PM must
see/check the end report, the Lead RSE or the person who asked for the grant to be in charge of getting the financial
report from F\&C and to send it to the external coordinator.
\end{itemize}
\item Closing of project
\begin{itemize}
\item Lead RSE informs PM of closing of project for funder or end of contract
\item PM follows formal closing steps as described by Section \ref{sec:closing}.
\end{itemize}
\end{enumerate}




\subsection{KD and SS projects}
The process for KD and SS projects is fully described by the KD and SS protocols~\cite{kd-intranet,ss-intranet}, respectively.
\begin{enumerate}[label=\arabic*.,ref=\arabic*]
\item Call and selection
\begin{itemize}
\item Follows the protocol
\item Upon granting, DoT (or TLs) gives F\&C instructions (cc'ed to PMs) which
projects are granted and need to be created in Exact.
\end{itemize}
\item Preparation of the project
\begin{itemize}
\item The PM formally assigns Lead RSE and discusses the planning with the Lead RSE and other
project team members and provides a planning in Ganttic.
\end{itemize}
\item During the project
\begin{itemize}
\item The PM is not directly involved in the project unless requested by Lead RSE.
\end{itemize}
\item Reporting
\begin{itemize}
\item The extent of reporting is decided upon by DoT and TLs.
\end{itemize}
\item Closing of project
\begin{itemize}
\item Lead RSE informs PM of closing of project
\item PM follows formal closing steps, as described in Section \ref{sec:closing}.
\end{itemize}
\end{enumerate}

\subsection{D\&C projects}
D\&C projects for external funding follow the same process as other external projects.

The process for the D\&C projects for training and workshops is as follows:
\begin{enumerate}[label=\arabic*),ref=\arabic*]
\item Each year in October the training lead from the D\&C team provides the SHs and PMs with an
overview of the hours needed for training purposes including an estimate of the hours needed for externally funded trainings
\item SHs discuss a planning with RSEs that will be involved in training in the coming year. The
total training hours should add up to the total number of hours needed for D\&C projects. SHs communicate the planning
to the PMs and to the training lead. PMs take the training planning into account in the projects planning.
\begin{enumerate}[label=\alph*.,ref=\alph*]
\item If the progress on projects that RSE is planned on seems to be at risk by training activities,
the PM signals this to the SH. Together they decide to either transfer the training hours to another RSE or to transfer
the project to another RSE.
\end{enumerate}
\item During the year
\begin{enumerate}[label=\alph*.,ref=\alph*]
\item If an RSE exceeds their agreed upon training hours, the SH is responsible for discussing this
issue with the RSE and taking appropriate measures.
\item If a project is (unexpectedly) progressing unsatisfactory due to the RSE involved in training,
the PM discusses this with the SH and the RSE and takes appropriate measures.
\item If the request for externally funded trainings exceeds the estimate given for that year, the
training lead contacts the PMs to discuss planning possibilities before agreeing to give this external training.
\end{enumerate}
\end{enumerate}


